\newpage
\section{Theoretical Background}
Power systems fundamentals
DC Optimal Power Flow explanation
Investment analysis concepts (NPV, annuities)
Mathematical formulation of the optimization problem
\subsection{Linear Programming in Energy Systems}
Overview of optimization techniques in energy asset management.

\subsection{DC Optimal Power Flow}
The DC Optimal Power Flow (DC-OPF) is a simplified version of the AC power flow problem, commonly used in power systems analysis and optimization~\cite{dc_opf}. This linearized model, thoroughly described in~\cite{andersson2004power}, makes the following key assumptions~\cite{wood2013power}:

\begin{itemize}
    \item Voltage magnitudes are assumed to be 1.0 per unit (p.u.)
    \item Line resistances are negligible compared to reactances ($R \ll X$)
    \item Voltage angle differences between connected buses are small
    \item Reactive power flows are ignored
\end{itemize}

\subsubsection{Mathematical Formulation}
The DCOPF problem is formulated as a linear programming optimization:

\begin{equation}
    \min_{\mathbf{P_g}, \boldsymbol{\theta}} \sum_{i \in \mathcal{G}} c_i(P_{g,i}) \label{eq:obj}
\end{equation}

Subject to the following constraints:

\textbf{Power Balance Constraints:}
\begin{equation}
    P_{g,i} - P_{d,i} = \sum_{j \in \mathcal{N}_i} B_{ij}(\theta_i - \theta_j) \quad \forall i \in \mathcal{N} \label{eq:power_balance}
\end{equation}

\textbf{Generation Limits:}
\begin{equation}
    P_{g,i}^{\min} \leq P_{g,i} \leq P_{g,i}^{\max} \quad \forall i \in \mathcal{G} \label{eq:gen_limits}
\end{equation}

\textbf{Line Capacity Limits:}
\begin{equation}
    -P_{ij}^{\max} \leq B_{ij}(\theta_i - \theta_j) \leq P_{ij}^{\max} \quad \forall (i,j) \in \mathcal{L} \label{eq:flow_limits}
\end{equation}

Where:
\begin{itemize}
    \item $P_{g,i}$: Power generation at bus $i$
    \item $P_{d,i}$: Power demand at bus $i$
    \item $\theta_i$: Voltage angle at bus $i$
    \item $x_{ij}$: Line reactance between buses $i$ and $j$
    \item $P_{ij}$: Power flow on line between buses $i$ and $j$
    \item $c_i$: Generation cost coefficient at bus $i$
\end{itemize}

\subsubsection{Additional Constraints for Investment Analysis}
For our investment optimization, we add the following constraints:

\textbf{Investment Budget Constraint:}
\begin{equation}
    \sum_{i \in G} I_i x_i \leq B_{max}
\end{equation}

\textbf{Capacity Factor Constraints:}
\begin{equation}
    \frac{1}{T} \sum_{t=1}^T \frac{P_{g,i,t}}{P_{g,i}^{max}} \geq CF_{min,i} \quad \forall i \in G
\end{equation}

Where:
\begin{itemize}
    \item $I_i$: Investment cost for generator $i$
    \item $x_i$: Binary decision variable for investment in generator $i$
    \item $B_{max}$: Maximum investment budget
    \item $CF_{min,i}$: Minimum required capacity factor for generator $i$
    \item $T$: Number of time periods
\end{itemize}

\subsubsection{Implementation Considerations}
\begin{itemize}
    \item The problem is implemented using Python with Pyomo optimization framework
    \item Seasonal variations are handled through multiple time periods
    \item Generator availability is modeled using capacity factors
    \item Investment decisions are binary variables
\end{itemize} 